\documentclass[a4paper]{article}

  \usepackage{amsmath}
  \usepackage{bbold}
  \usepackage{color}
  \usepackage{fancyvrb}
  \usepackage{graphicx}
  \usepackage{microtype}
  \usepackage{minted}
  \usepackage{parskip}
  \usepackage{tikz-cd}
  \usepackage{upquote}
  \usepackage{url}
  
  \usepackage[backend=biber,style=trad-abbrv]{biblatex}
  \usepackage[hidelinks,pdftex,unicode]{hyperref}
  
  \urlstyle{same}
  \UseMicrotypeSet[protrusion]{basicmath}
  
  \addbibresource{article.bib}
  
  \defbibenvironment{midbib}
    {\list
       {}
       {\setlength{\leftmargin}{\bibhang}%
        \setlength{\itemindent}{-\leftmargin}%
        \setlength{\itemsep}{\bibitemsep}%
        \setlength{\parsep}{\bibparsep}}}
    {\endlist}
    {\item}
  \title{GraphQL with Sangria}
  \author{Thomas Sutton}
  \date{31 October 2018}

\begin{document}

\maketitle

\begin{abstract}
  You're trapped in a maze of twisty distributed monolith REST APIs all alike.
  There is a small tribe of performance related Grues living in here. Maybe you
  should try GraphQL instead? It's still something of a twisty maze but at
  there are fewer Grues and you can see where you're going a bit more clearly.
\end{abstract}

\tableofcontents

\section{Introduction}

We've all heard about the on-going web application project which is implement as
a collection of microservices. Most of us are probably aware that most of these
services boil down to ``do REST to these database tables'' with the odd bit of
``and then poke that endpoint over there''. There are one or two things that
seem genuinely to benefit from the microservice architecture (the transcoding
support and, maybe, the real-time notification services) but the others it
doesn't really seem to get us much.

In the rest of this talk I'll sketch [some bits of] the existing REST interface
and point at some of its infelicities, and then talk about GraphQL and how a
GraphQL interface to this system might look.

\section{A sketch of a REST API}

profile/...
connection/...
feeds/...

\section{Denormalisation}

We can make some of this better by doing the sort of thing that gets called
``denormalisation'' in database land: embedding additional information into
our response/s so that clients won't need to make additional requests before
using the information. A good example is including some author profile
information into a list of feed posts:

\begin{minted}{json}
[
  {
    "id": "post:121947601",
    "author": {
      "id": "EA717EA7-DEAD-BEEF-DEAD-DEADBEEFBEEF",
      "name": "Thomas",
      "avatar": "https://images.example.com/delicious-cow.jpg"
    },
    "title": "My favourite cows",
    "body": "My top ten most delicious cows..."
  }
]
\end{minted}

This solves some of our problems -- we aren't stuck issuing $n+1$ REST queries
to get the information we need to display the a feed page -- but only for the
specific use case we consided when designing this denormalized view. If a client
requires some information that we didn't think to include they'll have to fall
back to $n+1$ queries.

\section{GraphQL}

Another option is to completely give up on representing objects of our model as
HTTP resources. Instead we can treat HTTP as a transport mechanism for our own
request/response protocols. The venerable XML-RPC and terrible SOAP protocols do
this but their both very general. Instead solutions like GraphQL address the
specific concern of modeling, querying and modifying graph-like data structures.

Conceptually, modelling a domain type for use in a GraphQL endpoint is pretty
similar to REST. The main difference is that we'll need to do some real work
for fields that represent relationships between different resources (the
``authorId'' field of a feed post, the ``profileId'' in a location, etc.)

We'll need to:

\begin{enumerate}
  \item support custom scalar types so we can keep our strong internal typing.

  \item describe types composed to scalar types.

  \item describe relationships that can be traversed using GraphQL.
  
  \item support parameterised queries so we can fetch data matching criteria
  supplied by the client.

  \item support updates so that clients can modify the data.
\end{enumerate}

\section{A sketch of a GraphQL schema}

I'm going to use Sangria -- a Scala GraphQL library -- here. It's pretty
straightforward, supports marshalling to all the JSON libraries that seem
reasonable to care about (and also jackson), and is the first solution I found
that seemed sensible to use.

\subsection{Object types}

\subsection{Relations}

\subsection{Updates}

\section{Implementation}

\begin{minted}{scala}
  val LocationType = ObjectType(
    "Location",
    "A physical location clients can attend.",
    List()
  )
\end{minted}

\subsection{Repository code}

\section{Bibliography}

\nocite{*}

\printbibliography[heading=none]

\end{document}