\documentclass[a4paper]{article}
\usepackage{amsmath}

\usepackage[backend=biber,style=numeric]{biblatex}
\addbibresource{ctt.bib}

\title{Some notes on Cubical Type Theory}
\author{Thomas Sutton}
\date{23 November 2017}

\newcommand{\fun}[3]{(#1 : #2) \rightarrow #3}
\newcommand{\labs}[2]{(\lambda #1 . #2)}
\newcommand{\lapp}[2]{(#1 \: #2)}

\newcommand{\pabs}[2]{{\langle #1 \rangle}.{#2}}
\newcommand{\papp}[2]{(#1 \: #2)}
\newcommand{\Path}[3]{{Path\:{#1}\:{#2}\:{#3}}}


\begin{document}

\maketitle

\section{Introduction}
I saw Ben Lippmeier give a short introduction to cubical type
theory\cite{DBLP:journals/corr/CohenCHM16} at the 100th FP-Syd meeting on 22 November
2017. Here are some thoughts.

\section{Extension equivalence of functions}

$\forall f g. (\forall x. f x = g x) \rightarrow f = g$

\section{Equality and equivalence}

\begin{itemize}
\item Syntactic
\item Definitional
\item
\item
\item Extensional
\end{itemize}

\section{A fragment of cubical type theory}

We have indexes with which we describe the ends of paths. There are
{\em not} first class parts of the language -- the only thing we can
do with them is feed them to path abstractions.

$$i ::= i | 1 | 0 $$

With indices defined we can describe the rest of the language. This
fragment is fairly standard as these things go, you can feel free to
assume whatever extensions along the lines of natural numbers, etc.

$$t,u,A,B ::= x | \labs{x}{t} | \lapp{t}{u} $$

$$\Path{A}{t}{u}$$
$$\pabs{i}{t}$$

\section{Function extensional}

We'll give a proof of the theorem $\forall f g. (\forall x. f x = g x)
\rightarrow f = g$ by translating it into a term of the appropriate
type. First we'll translate the $forall f g.$ into assumptions
asserting that they are functions of the appropriate type, then the
antecedant as an assumption of a function which constructs paths
equating applications of the two function, and finally the conclusion:
a term of the path type equating the $f$ and $g$.

\begin{align}
  \Gamma &\vdash f : \fun{x}{A}{B} \\
  \Gamma &\vdash g : \fun{x}{A}{B} \\
  \Gamma &\vdash p : \fun{x}{A}{\Path{B}{\lapp{f}{x}}{\lapp{g}{x}}} \\
  \cline{1-2}
  \Gamma &\vdash \pabs{i}{\labs{x}{
      \papp{\lapp{p}{x}}{i}
  }} : \Path{(\fun{x}{A}{B})}{f}{g}
\end{align}

We read this as: ``given f and g with the appropriate function type
and a function which, given $x$, constructs a path between $f x$ and
$g x$, we can produce a path between $f$ and $x$''. The path term
produced is not too hard to understand:

\begin{itemize}
  \item $\pabs{i}{t}$ because it's a path.
  \item $\pabs{i}{\labs{x}{t}}$ because it's between functions.
  \item $\pabs{i}{\labs{x}{\papp{\lapp{p}{x}}{i}}}$ because the value
    at $x$ is the value of the appropriate end of $p$.
\end{itemize}

\begin{itemize}
  \item It's a $Path$ but $\lapp{p}{t}$ is the wrong type so the
    outermost constructor must be $\pabs{i}{t}$.
  \item The $Path$ has functions as it's endpoints ($A =
    \fun{x}{A}{B}$) so body of the path abstraction is a function
    abstration: $\pabs{i}{\labs{x}{t}}$.
  \item The function returns the application of $\lapp{f}{x}$ at one
    end of the path and $\lapp{g}{x}$ at the other. The existing path
    $p$ does exactly this, so the body will construct a path using
    $\lapp{p}{x}$ and then pick the end we're interested in: $\pabs{i}{\labs{x}{\papp{\lapp{p}{x}}{i}}}$
\end{itemize}


\section{Whither primative paths}

The question that immediately occured to me during the talk is ``where
are the primitive paths?'' For example, other systems I'm familiar
with often have for reflexive equality of a term with itself, but
there are no such terms here.

After a little though later on it became obvious that a path from $t$
to $t$ is just a path abstraction which ignores the index:

$$
\frac{\Gamma \vdash t : A}{\Gamma \vdash \pabs{i}{t} : \Path{A}{t}{t}}
$$

\cite*{*}

\printbibliography
\end{document}
